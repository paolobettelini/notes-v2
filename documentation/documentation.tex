\documentclass[a4paper]{article}

\usepackage{amsmath}
\usepackage{amssymb}
\usepackage{hyperref}
\usepackage{lib}
\usepackage{fullpage}
\usepackage{dirtree}
\usepackage{xcolor}
\usepackage{parskip}

\hypersetup{
    colorlinks=true,
    linkcolor=black,
    urlcolor=blue,
    pdftitle={Stellar - Documentazione},
    pdfpagemode=FullScreen,
}

\title{%
    Stellar \\
    \phantom{} \\
    \Large Courseware environment \\
    \Large Documentation
}

\author{Paolo Bettelini}

\date{}

\begin{document}

\maketitle
\tableofcontents
\pagebreak

\part{Stellar}

\section{Introduction}

\section{Data structure}

\dirtree{%
.1 data/.
.2 snippets/.
.2 pages/.
.2 courses/.
.2 universes/.
}

\pagebreak

\section{PDF Stellar Format}

\subsection{Format commands}

Stellar is able to interpret special PDFs with certain text commands written in the page.
This is referred to as \texttt{PDF Stellar Format}.
The file represents a Stellar page containing snippets, or just a collection of snippets.
Stellar will crop the PDF into multiple pieces to generate every snippet individually,
and is also able to generate the corresponding page containing them.

\bgroup{}
\def\arraystretch{1.5}
\begin{center}
    \begin{tabular}{ |p{4.5cm}|p{8cm}| }
        \hline
        \multicolumn{2}{|c|}{\textbf{PDF Stellar Format commands}} \\
        \hline
        \textbf{Command} & \textbf{Description} \\
        \hline
        \texttt{!id <ID>} & Set the ID of this page. \\
        \hline
        \texttt{!snippet <ID>} & Start a PDF snippet here. \\
        \hline
        \texttt{!endsnippet <JSON meta>} & End the current snippet here. The JSON metadata is optional. \\
        \hline
        \texttt{!gen-page <bool>} & Set to true to generate the HTML page for this PDF. \\
        \hline
        \texttt{!include <ID> <Params>} & Include a snippet here given its ID. The parameters string is optional. \\
        \hline
        \texttt{!plain} & Write plain HTML to the page here. \\
        \hline
        \texttt{!section} & Add a level 1 heading \texttt{<h1>}. \\
        \hline
        \texttt{!subsection} & Add a level 2 heading \texttt{<h2>}. \\
        \hline
        \texttt{!subsubsection} & Add a level 3 heading \texttt{<h3>}. \\
        \hline
    \end{tabular}
\end{center}
\egroup{}

Furthermore, it is possible to reference another snippet by adding an annotation
with a link of the form \texttt{/snippet/snippet-id} or \texttt{/snippet/snippet-id|Label}.

Stellar will crop the PDF between \texttt{!snippet} amd \texttt{!endsnippet}.

\pagebreak

\subsection{\LaTeX Package}

The file \texttt{stellar.sty} provides the \LaTeX package \texttt{stellar}, which
implements PDF Stellar Format.

\subsubsection{Usage}

In order to use the package is it necessary to use the following documentclass:
\begin{lstlisting}[style=boxed, style=LaTeX]
    \documentclass[preview]{standalone}
\end{lstlisting}
and import the \texttt{stellar} package
\begin{lstlisting}[style=boxed, style=LaTeX]
    \usepackage{stellar}
\end{lstlisting}

The command \lstinline[style=LaTeX]{\title} needs to be inserted within the \texttt{document}.
The commands \lstinline[style=LaTeX]{\section}, \lstinline[style=LaTeX]{\subsection} and
\lstinline[style=LaTeX]{\subsubsection} can be used normally as their behavior is overwritten by the package.

The commands \lstinline[style=LaTeX]{\id}, \lstinline[style=LaTeX]{\genpage}
and \lstinline[style=LaTeX]{\plain} are available.

It is possible to reference another snippet by using the \lstinline[style=LaTeX]{\snippetref} command:
\begin{lstlisting}[style=boxed, style=LaTeX]
    \snippetref[snippet-identifier][Some text].
\end{lstlisting}

A snippet can be included using the following command:
\begin{lstlisting}[style=boxed, style=LaTeX]
    \includesnpt{snippet-identifier}
    \includesnpt[param1=value1|param2=value2]{snippet-identifier}
\end{lstlisting}

Snippets can be created using environments. A plain snippet can be created as follows:
\begin{lstlisting}[style=boxed, style=LaTeX]
    \begin{snippet}{snippet-identifier}
        % snippet content
    \end{snippet}
\end{lstlisting}

There are other types of snippets such as the following

\begin{lstlisting}[style=boxed, style=LaTeX]
    \begin{snippetdefinition}{snippet-definition1}{Definition 1}
        % snippet content
    \end{snippetdefinition}

    \begin{snippettheorem}{snippet-theorem1}{Theorem 1}
        % snippet content
    \end{snippettheorem}

    \begin{snippetproof}{snippet-proof1}{snippet-theorem1}{Proof 1}
        % snippet content
    \end{snippetproof}
\end{lstlisting}

\pagebreak

\subsubsection{Example}

\begin{lstlisting}[style=boxed, style=LaTeX]
\documentclass[preview]{standalone}

\usepackage{stellar}

\begin{document}

\id{page-identifier}
\genpage

\section{Section 1}

\subsection{Subsection 1}

\begin{snippetdefinition}{snippet1-definition}{Some definition}
    This is some definition
\end{snippetdefinition}

\subsection{Subsection 2}

\section{Section 2}

\begin{snippetdefinition}{snippet2-definition}{Some definition}
    This is some definition, the sequel
\end{snippetdefinition}

\includesnpt{snippet2-definition}

\begin{snippet}{text-snippet}[\{"meta": "json"\}]
    Some text as a snippet!
\end{snippet}

\end{document}
\end{lstlisting}

\pagebreak

\section{Metadata}

A snippet may contain a \texttt{metadata.json} file containing some metadata properties.

\subsection{JSON structure}

\bgroup{}
\def\arraystretch{1.5}
\begin{center}
    \begin{tabular}{ |p{3cm}|p{4cm}|p{5cm}| }
        \hline
        \multicolumn{3}{|c|}{\textbf{Metadata fields}} \\
        \hline
        \textbf{Field} & \textbf{Type} & \textbf{Description} \\
        \hline
        \texttt{generalizations} & Array of snippet IDs & Mathematical generalizations. \\
        \hline
        \texttt{requires} & Array of snippet IDs & Necessary libraries. \\
        \hline
        \texttt{default-params} & Params string & Default parameters for snippet. \\
        \hline
    \end{tabular}
\end{center}
\egroup{}

\subsection{Example}

\begin{lstlisting}[style=boxed, style=json]
    {
        "default-params": "width=70%|src=https://youtu.be/dQw4w9WgXcQ&",
        "requires": ["some-lib-snippet"],
    }
\end{lstlisting}

\pagebreak

\part{\LaTeX-driven implementation}

\section{Structure}

\dirtree{%
.1 source/.
.2 latex/.
.2 courses/.
.2 pages/.
.2 snippets/.
.2 universes/.
}

\pagebreak

\section{Snippet build scripts}

By default, snippets in the \texttt{snippets/} folder are just copied to the \texttt{data/} folder by the compiler.
However, it is possible to add a custom build script \texttt{build.py} to the snippet directory to implement
custom behavior.
The build script will receive a single parameter representing the target folder where the files need to be written to.

\subsection{Example}

The following is an example build script to compile a Rust \href{https://github.com/nannou-org/nannou}{nannou} project to wasm.

\begin{lstlisting}[style=boxed, style=Python]
import sys
import subprocess
import os
import shutil

# This build script compiles a nannou project into a snippet

def main():
    target_folder = sys.argv[1]
    
    try:
        # Execute wasm-pack build --release
        subprocess.run(["wasm-pack", "build", "--release"], check=True)
        
        # Change directory to website
        os.chdir("website")
        
        # Execute npm install
        subprocess.run(["npm", "install"], check=True)
        
        # Execute npm run build
        subprocess.run(["npm", "run", "build"], check=True)
        
        # Generate snippet folder
        dist_folder = "dist"
        target_files = os.listdir(dist_folder)
        
        # Remove "dist/index.js"
        index_file_path = os.path.join(dist_folder, "index.js")
        if os.path.exists(index_file_path):
            os.remove(index_file_path)
        
        # Move all files from "dist" to target_folder
        for file_name in target_files:
            file_path = os.path.join(dist_folder, file_name)
            if os.path.isfile(file_path):
                shutil.move(file_path, target_folder)
        
        # Remove the "dist" directory
        shutil.rmtree(dist_folder, ignore_errors=True)
        
        print(f"Successfully moved files to {target_folder}")
    
    except subprocess.CalledProcessError as e:
        print(f"An error occurred while executing: {e.cmd}")
        print(f"Return code: {e.returncode}")
        sys.exit(1)
    except Exception as e:
        print(f"An unexpected error occurred: {str(e)}")
        sys.exit(1)

    if __name__ == "__main__":
        main()
\end{lstlisting}

\end{document}