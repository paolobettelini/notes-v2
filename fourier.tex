\documentclass{article}

\usepackage{amsmath}
\usepackage{amssymb}
\usepackage{parskip}
\usepackage{hyperref}
\usepackage{xcolor}
\usepackage[active, tightpage]{preview}
\usepackage[pass,paperwidth=20cm]{geometry}

\pagenumbering{gobble}

\hypersetup{
    colorlinks=true,
    linkcolor=black,
    urlcolor=blue,
    pdftitle={Test},
    pdfpagemode=FullScreen,
}

\newcommand{\integral}[4]{\int\limits_{#1}^{#2} #3\,d#4}
    
\begin{document}
\begin{preview}
\pagecolor[HTML]{161923}
\color[HTML]{c8c9db}
\Huge


Let's look at a simple example. We are going to derive the Fourier series of a function \(f(x)\) defined as such:

\[
    f(x)=
    \begin{cases}
        -1\quad \text{if } -\pi < x < 0 \\
        +1\quad \text{if } 0 < x < \pi \\
    \end{cases}
\]

The period of this function is \(T=2\pi\). We can already simplify the \(\frac{2\pi}{T}\) term, leaving us with

\[
    f(x)=\frac{a_0}{2} + \sum_{n=1}^{\infty} a_n \cos(nx) + b_n \sin(nx)
\]

First, we need to find \(a_n\). Simplifying \(\frac{2\pi}{T}\) and \(\frac{T}{2}\) we get

\[
    a_n=\integral{-\pi}{\pi}{f(x)\cos(nx)}{x}
\]

Looking at the graph we notice that we can split the integral into two parts at \(x=0\).
On the left part, the function is \(-\cos(nx)\), while on the right part the function is \(\cos(nx)\).

\begin{align*}
    a_n &=
    \frac{1}{\pi} \integral{-\pi}{0}{-\cos(nx)}{x} +
    \frac{1}{\pi} \integral{0}{\pi}{\cos(nx)}{x} \\
    &= -\frac{1}{\pi} \integral{-\pi}{0}{\cos(nx)}{x} +
    \frac{1}{\pi} \integral{0}{\pi}{\cos(nx)}{x} \\
    &= -\frac{1}{\pi} {\left[\frac{\sin(xn)}{n}\right]}_{-\pi}^{0} +
    \frac{1}{\pi} {\left[\frac{\sin(xn)}{n}\right]}_{0}^{-\pi} \\
    &= -\frac{1}{\pi} \left[\frac{\sin(\pi n)}{n}\right] +
    \frac{1}{\pi} \left[\frac{\sin(\pi n)}{n}\right] \\
    &= \left(\frac{1}{\pi}-\frac{1}{\pi}\right) \left[\frac{\sin(\pi n)}{n}\right] \\
    &= 0
\end{align*}

\(a_n\) is always going to be 0. (note). We can remove the \(a_n \cos(nx)\) and \(\frac{a_0}{2}\) terms from the series.

Now for \(b_n\)

\[
    b_n=\integral{-\pi}{\pi}{f(x)\sin(nx)}{x}
\]

Again, we split the integral into two parts

\begin{align*}
    b_n &=
    -\frac{1}{\pi} \integral{-\pi}{0}{\sin(nx)}{x} +
    \frac{1}{\pi} \integral{0}{\pi}{\sin(nx)}{x} \\
    &= -\frac{1}{\pi} {\left[\frac{-\cos(xn)}{n}\right]}_{-\pi}^{0} +
    \frac{1}{\pi} {\left[\frac{-\cos(xn)}{n}\right]}_{0}^{-\pi} \\
    &= -\frac{1}{\pi} \left[-\frac{1}{n}+\frac{\cos(\pi n)}{n}\right] +
    \frac{1}{\pi} \left[\frac{-\cos(\pi n)}{n}+\frac{1}{n}\right] \\
    &= -\frac{1}{\pi} \left[\frac{\cos(\pi n)-1}{n}\right] +
    \frac{1}{\pi} \left[\frac{1-\cos(\pi n)}{n}\right] \\
    &= \frac{2}{\pi} \cdot \frac{1-\cos(\pi n)}{n} \\
    &= \frac{2-2\cos(\pi n)}{\pi n}
\end{align*}

\end{preview}
\end{document}